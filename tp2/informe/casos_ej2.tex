\subsection{Casos de prueba}

Para este ejercicio pensamos varias instancias de prueba como las siguientes:

\begin{itemize}
\item Caso de un tablero de 2 x 2 con dos caballos ubicados en distintos casilleros. Este caso no tiene solucion, ya que el tablero no tiene un tamaño suficiente como para que alguno de los caballos realice un movimiento.
\item Caso de un tablero de (ver despues el tamaño) con 4 caballos ubicados en cada una de las puntas del tablero. Aquí la solución tiene que dar el casillero que se encuentra justo en el medio del tablero.
\item Caso de un tablero (en el que no importa el tamaño) con un solo caballo. En este caso como solo hay un caballo, no seria necesario mover el caballo ya que es el único del tablero.
\item Caso 3 con 2 caballos. Este caso es igual al caso anterior, solo que hay dos caballos en la misma casilla. Al igaul que en el caso anterior, la optimicidad se obtiene no realizando ningun movimiento, osea, con una cantidad de movimientos igual a cero.
\item
\end{itemize}