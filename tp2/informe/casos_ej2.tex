\subsection{Casos de prueba}

Para este ejercicio pensamos varias instancias de prueba que nos sirvieron para testear si el algoritmo estaba resolviendo bien lo pretendido como tambien para divertirnos un poco tratando de intuir por donde iba a ser el casillero de menor cantidad de movimientos en relacion a la disposici\'on de los caballos.

\begin{itemize}
\item Caso de un tablero de 2 x 2 con dos caballos ubicados en distintos casilleros. Este caso no tiene solucion, ya que el tablero no tiene un tamaño suficiente como para que alguno de los caballos realice un movimiento.
\item Caso de un tablero de 5 x 5 con 8 caballos ubicados en los extremos a los cuales se podria mover otro caballo ubicado el el casillero centarl del tablero. Aquí la solución tiene que dar el casillero que se encuentra justo en el medio del tablero con 8 movimientos (uno por cada caballo).
\item Caso de un tablero (en el que no importa el tamaño) con un solo caballo. En este caso como solo hay un caballo, no seria necesario mover el caballo ya que es el único del tablero.
\item Caso 3 con un numero arbitrario de caballos. Este caso es igual al caso anterior, solo que hay dos caballos en la misma casilla. Al igaul que en el caso anterior, la optimicidad se obtiene no realizando ningun movimiento, osea, con una cantidad de movimientos igual a cero.
\item Caso de un tablero con caballos en todas sus posiciones.
\item Case de un tablero con caballos en posiciones aleatoreas.
\end{itemize}