\subsection{Descripci\'on del problema}

Tenemos un juego de un solo jugador, el cual consta de un tablero como el de ajedrez, con la diferencia de que el tablero tiene un tamaño de \texttt{N}x\texttt{N} casillas, y las piezas con las que se juegan son solo caballos. El juego comienza con una cantidad \texttt{K} de caballos blancos repartidos, sin ninguna particularidad, por todo el tablero. Y el objetivo del juego es encontrar (en caso de que el juego tenga solucion), un casillero donde poder ubicar todos los caballos del tablero, habiendo hecho la menor cantidad de movimientos posibles. 
Algunas aclaraciones:
\begin{itemize}
\item El movimiento de los caballos es igual al movimiento de los mismos en el juego de ajedrez, en forma de "L", la imagen 1.
\item En todo momento puede haber mas de un caballo en la misma casilla del tablero. El juego podria empezar con mas de un caballo en alguna casilla.
\item Un movimiento se cuenta como el movimiento de un caballo de una casilla a otra.
\end{itemize}

\begin{figure}[H]
\begin{center}
\includegraphics[scale=.45]{./imagenes/ej2_movimientoCaballo.png}
\caption{Img2: Movimientos posibles de un caballo en el tablero.}
\end{center}
\end{figure}

\newpage