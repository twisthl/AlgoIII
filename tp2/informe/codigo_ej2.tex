\subsection{C\'odigo fuente}

El siguiente es el c\'odigo relevante del ejercicio. Se omiten varias partes que no hacen a la resoluci\'on general. Se pueden conciderar importantes el modelado del grafo y el BFS.

\subsubsection{Tipos}
Los siguientes son los tipos utilizados para facilitar la manipulaci\'on de datos en el problema.

\lstset{language=C++,
                basicstyle=\ttfamily\footnotesize,
                keywordstyle=\color{blue}\ttfamily,
                stringstyle=\color{red}\ttfamily,
                commentstyle=\color{green}\ttfamily,
                morecomment=[l][\color{magenta}]{\#},
                breaklines=true
}
\begin{lstlisting}


struct Casillero{
	int x;
	int y;
};

typedef std::list<Casillero> ListaAdyacencia;
typedef std::vector<Casillero> Caballos;

struct Nodo{
	int x;
	int y;
	int movimientosNecesarios;
	int caballosLlegaron;
	ListaAdyacencia adyacentes;
	int distancia;
};

typedef std::vector< vector<Nodo> > MatrizDeNodos;

struct Solucion{
	int x;
	int y;
	int movimientos;
};

\end{lstlisting}

Los Casillero se utilizan tanto para representar caballos, como casilleros en la lista de adyacentes de cada nodo.

Los nodos se inicializan con su posicion, 0 movimientos necesarios, 0 caballos que lo alcanzan y con su lista de adyacentes vacia, la cual va a ser calculada en el modelado. El atributo distancia de Nodo se utiliza en cada BFS para llevar la cantidad de movimientos que necesita un caballo para llegar a ese nodo (la cantidad de movimientos de su antecesor + 1).

\subsubsection{Modelado}

En este m\'etodo' es donde se crea la matriz de \texttt(N) $x$ \texttt(N) nodos inicializados y se comienzan a cargar los casilleros adyacentes que les correspondan.
De las 8 posibilidades de vecinos que tiene cualquier casillero, solo se agregan aquellas que repesenten un casillero v\'alido dentro de la matriz, esto quiere decir que sus coordenadas \texttt(X) e \texttt(Y) sean $<= 0$ y $< N$.

\lstset{language=C++,
                basicstyle=\ttfamily\footnotesize,
                keywordstyle=\color{blue}\ttfamily,
                stringstyle=\color{red}\ttfamily,
                commentstyle=\color{green}\ttfamily,
                morecomment=[l][\color{magenta}]{\#},
                breaklines=true
}
\begin{lstlisting}

MatrizDeNodos* modelar(int n){
  vector<Nodo> aux(n);
  MatrizDeNodos* matriz = new MatrizDeNodos(n, aux);

  for(int i = 0; i < n; i++){
    for(int j = 0; j < n; j++){
      Nodo nodo(i,j);
      if( 0<= i+2 && i+2 < n){
        if ( 0 <= j+1 && j+1 < n){
          nodo.adyacentes.push_back(Casillero(i+2,j+1));
        }
        if (0 <= j-1 && j-1 < n){
          nodo.adyacentes.push_back(Casillero(i+2,j-1));
        }
      }
      if(0 <= i-2 && i-2 < n){
        if (0 <= j+1 && j+1 < n){
          nodo.adyacentes.push_back(Casillero(i-2,j+1));
        }
        if (0 <= j-1 && j-1 < n){
          nodo.adyacentes.push_back(Casillero(i-2,j-1));
        }
      }
      if(0 <= j+2 && j+2 < n){
        if (0 <= i+1 && i+1 < n){
          nodo.adyacentes.push_back(Casillero(i+1,j+2));
        }
        if (0 <= i-1 && i-1 < n){
          nodo.adyacentes.push_back(Casillero(i-1,j+2));
        }
      }
      if(0 <= j-2 && j-2 < n){
        if (0 <= i+1 && i+1 < n){
          nodo.adyacentes.push_back(Casillero(i+1,j-2));
        }
        if (0 <= i-1 && i-1 < n){
          nodo.adyacentes.push_back(Casillero(i-1,j-2));
        }
      }
      (*matriz)[i][j] = nodo;
    }
  }
  return matriz;
}

\end{lstlisting}

\subsubsection{BFS}

\lstset{language=C++,
                basicstyle=\ttfamily\footnotesize,
                keywordstyle=\color{blue}\ttfamily,
                stringstyle=\color{red}\ttfamily,
                commentstyle=\color{green}\ttfamily,
                morecomment=[l][\color{magenta}]{\#},
                breaklines=true
}
\begin{lstlisting}

void bfs(MatrizDeNodos& matriz, int n, Casillero& c){
	\\ Para no repetir nodos me guardo otra matriz de visitados
	std::vector<bool> aux(n,false);
	std::vector< vector<bool> > visitado(n,aux);
	for(int i = 0; i < n; i++){
		for(int j = 0; j < n; j++){
		visitado[i][j] = false;
		}
	}

	\\ - 1 pues la matiz va del (0,0) y los caballos (1,1)
	int x, y;
	x = c.x -1;
	y = c.y -1;

	\\La posicion de donde parte el caballo
	\\Lo visite
	visitado[x][y] = true;
	\\A este nodo lo alcanzo este caballo
	matriz[x][y].caballosLlegaron++;
	\\La distancia para el BFS de este caballo al nodo origen es 0
	matriz[x][y].distancia = 0;

	std::queue<Nodo> cola;

	\\Recorro sus adyacentes inmediatos
	ListaAdyacencia::iterator itPrimeros = matriz[x][y].adyacentes.begin();
	for(; itPrimeros != matriz[x][y].adyacentes.end(); ++itPrimeros){
		Casillero casillero = *itPrimeros;
		int a = casillero.x;
		int b = casillero.y;
		visitado[a][b] = true;
		matriz[a][b].distancia = 1;
		matriz[a][b].caballosLlegaron++;
		matriz[a][b].movimientosNecesarios += 1;

		\\Los meto en un cola para recorreluego sus hijos
		cola.push(matriz[a][b]);
	}

	\\Recorro los elementos encolados y al mismo tiempo agrego elementos que no visite
	while(!cola.empty()){
		\\Saco el primero
		Nodo nodo = cola.front();
		cola.pop();

		\\Recorro sus adyacentes
		ListaAdyacencia::iterator itAdyacentes = nodo.adyacentes.begin();
		for(; itAdyacentes != nodo.adyacentes.end(); ++itAdyacentes){
			Casillero casillero = *itAdyacentes;
			int a = casillero.x;
			int b = casillero.y;
			\\Si no lo visite lo marco, seteo sus valores y lo agrego a la cola
			if(!visitado[a][b]){
				visitado[a][b] = true;
				matriz[a][b].distancia = nodo.distancia + 1;
				matriz[a][b].caballosLlegaron++;
				matriz[a][b].movimientosNecesarios += matriz[a][b].distancia;

				cola.push(matriz[a][b]);
			}
		}
	}
}

\end{lstlisting}