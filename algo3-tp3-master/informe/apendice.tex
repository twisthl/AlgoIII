\section{Apéndice: Códigos fuente}

\subsection{Algoritmo exacto}

\lstset{language=C++,
        keywordstyle=\color{blue},
        stringstyle=\color{red},
        commentstyle=\color{magenta},
        morecomment=[l][\color{magenta}]{\#}
}
\begin{lstlisting}[frame=single]
void kpmp(vector<vector<double> > &adym, Partition &partition, Partition &min_partition, int node, map<string, bool> &options)
{
  // check if has added all nodes
  if (node == adym.size()) {
    if (partition.weight < min_partition.weight &&
       partition.subsets_used <= partition.allowed_subsets) {
      min_partition = partition;
    }
    return;
  }

  // poda k_subsets
  if (options["k_subsets"]) {
    int subsets_left = partition.allowed_subsets - partition.subsets_used;
    if (nodes_left(adym.size(), node) == subsets_left) {
      partition.push_back_to_subset(partition.subsets_used, node);
      kpmp(adym, partition, min_partition, node+1, options);
      partition.pop_back_from_subset(partition.subsets_used-1);
    }
  }

  // poda pesada
  if (options["can_improve"]) {
    if (partition.subsets_used == partition.allowed_subsets) {
      double min_weight_of_adding_rest = 0;
      for (int current_node = node; current_node < adym.size(); current_node++) {
        double min_weight_in_subset = numeric_limits<double>::max();
        for (int subset = 0; subset < partition.subsets_used; ++subset) {
          double current_weight = partition.weight_in_subset(adym, current_node, subset);
          if (current_weight < min_weight_in_subset) {
            min_weight_in_subset = current_weight;
          }
        }
        min_weight_of_adding_rest += min_weight_in_subset;
      }

      if (min_weight_of_adding_rest + partition.weight >= min_partition.weight) {
        return;
      }
    }
  }

  // try adding it to every non-empty subset
  for (int subset = 0; subset < partition.subsets_used; ++subset) {
    double added_weight = partition.weight_in_subset(adym, node, subset);
    // poda min_weight
    if (options["min_weight"]) {
      if (partition.weight+added_weight < min_partition.weight) {
        partition.push_back_to_subset(subset, node);
        partition.weight += added_weight;
        kpmp(adym, partition, min_partition, node+1, options);
        partition.pop_back_from_subset(subset);
        partition.weight -= added_weight;
      }
    } else {
      partition.push_back_to_subset(subset, node);
      partition.weight += added_weight;
      kpmp(adym, partition, min_partition, node+1, options);
      partition.pop_back_from_subset(subset);
      partition.weight -= added_weight;
    }
  }

  if (options["k_subsets"]) {
    if (partition.subsets_used < partition.allowed_subsets) {
      // try adding new partition
      partition.push_back_to_subset(partition.subsets_used, node);
      kpmp(adym, partition, min_partition, node+1, options);
      partition.pop_back_from_subset(partition.subsets_used-1);
    }
  } else {
    if (partition.subsets_used < partition.total_subsets()) {
      partition.push_back_to_subset(partition.subsets_used, node);
      kpmp(adym, partition, min_partition, node+1, options);
      partition.pop_back_from_subset(partition.subsets_used-1);
    }
  }
}
\end{lstlisting}

\newpage

\subsection{Heurística golosa constructiva (pura)}

\lstset{language=C++,
        keywordstyle=\color{blue},
        stringstyle=\color{red},
        commentstyle=\color{magenta},
        morecomment=[l][\color{magenta}]{\#}
}
\begin{lstlisting}[frame=single]
std::vector<std::set<int>> Heuristica::resolverGolosoPuro() {
    std::vector<std::set<int>> res(k_);
    int n = grafo_.getCantidadVertices();
    if (n <= k_) {
        for (int i = 0; i < n; i++) {
            res[i].insert(i);
        }
    } else {
        std::vector<int> verticesOrdenadosPorPeso = ordenarPorPesoEnGrafo();
        for (auto & v : verticesOrdenadosPorPeso) {
            double mejorPeso = pesoEnSubconjunto(v, res[0]);
            int mejorSubconjunto = 0;
            for (int i = 1; i < k_; i++) {
                int pesoEnSubconj = pesoEnSubconjunto(v, res[i]);
                if ( pesoEnSubconj < mejorPeso ) {
                    mejorPeso = pesoEnSubconj;
                    mejorSubconjunto = i;
                }
            }
            res[mejorSubconjunto].insert(v);
        }
    }
    return res;
}
\end{lstlisting}

\newpage

\subsection{Heurística golosa constructiva aleatorizada}

\lstset{language=C++,
        keywordstyle=\color{blue},
        stringstyle=\color{red},
        commentstyle=\color{magenta},
        morecomment=[l][\color{magenta}]{\#}
}
\begin{lstlisting}[frame=single]
std::vector<std::set<int>> Heuristica::resolver() {
    std::vector<std::set<int>> res(k_);
    std::vector<int> verticesOrdenadosPorPeso = ordenarPorPesoEnGrafo();
    int cantidadConjuntosCandidatos = std::min(k_, profundidadEleccionConjunto_);
    while (!verticesOrdenadosPorPeso.empty()) {
        int cantidadVerticesCandidatos = std::min((int)verticesOrdenadosPorPeso.size(), profundidadEleccionVertice_);
        int indicePorRemover = rand() % cantidadVerticesCandidatos;
        int verticeNuevo = verticesOrdenadosPorPeso[indicePorRemover];
        verticesOrdenadosPorPeso.erase(verticesOrdenadosPorPeso.begin() + indicePorRemover);
        std::vector<std::pair<int,double>> infoConjuntos; // Esto guarda las tuplas <conjunto i, peso del vertice en conjunto i>
        for (int i = 0; i < k_; i++) {
            infoConjuntos.push_back(std::make_pair(i, pesoEnSubconjunto(verticeNuevo, res[i])));
        }
        std::sort(infoConjuntos.begin(), infoConjuntos.end(), [] (const std::pair<int,double> & a, const std::pair<int,double> & b) { return a.second < b.second; });
        res[infoConjuntos[rand() % cantidadConjuntosCandidatos].first].insert(verticeNuevo);
    }
    return res;
}
\end{lstlisting}

\newpage

\subsection{Heurística de búsqueda local con vecindad (A)}

\lstset{language=C++,
        keywordstyle=\color{blue},
        stringstyle=\color{red},
        commentstyle=\color{magenta},
        morecomment=[l][\color{magenta}]{\#}
}
\begin{lstlisting}[frame=single]
int main(int argc, char *argv[])
{
  int n, m, k;
  cin >> n >> m >> k;

  // having w((u, v)) = 0 || (u, v) not in E is the same 
  vector<vector<double> > adym(n, vector<double> (n, 0));
  int u, v;
  double w;
  double total_weight = 0;
  for(int i = 0; i < m; i++) {
    cin >> u >> v >> w;
    adym[u][v] = w;
    adym[v][u] = w;
    total_weight += w;
  }

  vector<set<int> > partition(k, set<int>());
  for(int i = 0; i < n; ++i) {
    partition[0].insert(i);
  }

  cout << "Starting with a total weight of " << total_weight << endl;
  vector<int> node_indexed_partition(n, 0);
  bool has_improved = true;
  while (has_improved) {
    has_improved = false;
    for (int i = 0; i < n; ++i) {
      double node_weight_in_current_subset = 
        node_weight_in_subset(i, node_indexed_partition[i], partition, adym);
      bool swapped = false;
      int subset = 0;
      while (!swapped && subset < k) {
        if (subset != node_indexed_partition[i]) {
          cout << "Considering swapping node " << i << " from subset " << node_indexed_partition[i] << " to subset " << subset << endl;
          double node_weight_in_subset_j = node_weight_in_subset(i, subset, partition, adym);
          if (node_weight_in_current_subset > node_weight_in_subset_j) {
            cout << "FOUND IMPROVEMENT!" << endl;
            cout << " Swapping node " << i << " from subset " << node_indexed_partition[i] << " to subset " << subset << endl;
            partition[node_indexed_partition[i]].erase(i);
            partition[subset].insert(i);
            node_indexed_partition[i] = subset;
            total_weight = total_weight - node_weight_in_current_subset + node_weight_in_subset_j;
            cout << "New total weight is: " << total_weight << endl;
            has_improved = true;
            swapped = true;
          }
        }
        ++subset;
      }
    }
  }

  cout << "Minimum weight reached: " << total_weight << endl;
  cout << "PARTITION" << endl;
  for (int i = 0; i < k; ++i) {
    cout << "Partition " << i << ": ";
    for(set<int>::iterator it = partition[i].begin(); it != partition[i].end(); ++it) {
      cout << *it << ' ';
    }
    cout << endl;
  }

  return 0;
}
\end{lstlisting}

\newpage

\subsection{Heurística de búsqueda local con vecindad (B)}

\lstset{language=C++,
        keywordstyle=\color{blue},
        stringstyle=\color{red},
        commentstyle=\color{magenta},
        morecomment=[l][\color{magenta}]{\#}
}
\begin{lstlisting}[frame=single, breaklines]
int main(int argc, char *argv[])
{
  int n, m, k;
  cin >> n >> m >> k;

  // having w((u, v)) = 0 || (u, v) not in E is the same 
  vector<vector<double> > adym(n, vector<double> (n, 0));
  int u, v;
  double w;
  double total_weight = 0;
  for(int i = 0; i < m; i++) {
    cin >> u >> v >> w;
    adym[u][v] = w;
    adym[v][u] = w;
    total_weight += w;
  }

  vector<set<int> > partition(k, set<int>());
  for(int i = 0; i < n; ++i) {
    partition[0].insert(i);
  }

  cout << "Starting with a total weight of " << total_weight << endl;
  vector<int> node_indexed_partition(n, 0);
  bool has_improved = true;
  while (has_improved) {
    has_improved = false;
    for (int first_node = 0; first_node < n; ++first_node) {
      double first_node_current_weight = node_weight_in_subset(first_node, node_indexed_partition[first_node], partition, adym, -1);
      bool swapped = false;
      int second_node = first_node+1;
      while (!swapped && second_node < n) {
        if (node_indexed_partition[second_node] != node_indexed_partition[first_node]) {
          cout << "Considering swapping node " << first_node << " from subset " << node_indexed_partition[first_node] << 
            " for node" << second_node << " on subset " << node_indexed_partition[second_node] << endl;
          double second_node_current_weight = node_weight_in_subset(second_node, node_indexed_partition[second_node], partition, adym, -1);
          double first_node_new_weight = node_weight_in_subset(first_node, node_indexed_partition[second_node], partition, adym, second_node);
          double second_node_new_weight = node_weight_in_subset(second_node, node_indexed_partition[first_node], partition, adym, first_node);
          // if what I remove is greater than what I'm adding
          bool swap_improves = (first_node_current_weight+second_node_current_weight) > (first_node_new_weight+second_node_new_weight);
          if (swap_improves) {
            cout << "FOUND IMPROVEMENT!" << endl;
            cout << " Swapping node " << first_node << " from subset " << node_indexed_partition[first_node] << "for node " << 
              second_node << " on subset " << node_indexed_partition[second_node] << endl;
            partition[node_indexed_partition[first_node]].erase(first_node);
            partition[node_indexed_partition[second_node]].erase(second_node);
            partition[node_indexed_partition[first_node]].insert(second_node);
            partition[node_indexed_partition[second_node]].insert(first_node);
            int temp = node_indexed_partition[second_node];
            node_indexed_partition[second_node] = node_indexed_partition[first_node];
            node_indexed_partition[first_node] = temp;
            total_weight = total_weight - (first_node_current_weight+second_node_current_weight) + (first_node_new_weight+second_node_new_weight);
            cout << "New total weight is: " << total_weight << endl;
            has_improved = true;
            swapped = true;
          }
        }
        ++second_node;
      }
    }
  }

  cout << "Minimum weight reached: " << total_weight << endl;
  cout << "PARTITION" << endl;
  for (int i = 0; i < k; ++i) {
    cout << "Partition " << i << ": ";
    for(set<int>::iterator it = partition[i].begin(); it != partition[i].end(); ++it) {
      cout << *it << ' ';
    }
    cout << endl;
  }

  return 0;
}
\end{lstlisting}

\newpage

\subsection{Metaheurística GRASP}

\lstset{language=C++,
        keywordstyle=\color{blue},
        stringstyle=\color{red},
        commentstyle=\color{magenta},
        morecomment=[l][\color{magenta}]{\#}
}
\begin{lstlisting}[frame=single]
double Grasp::ejecutar(int criterioParada) { // Ejecuta GRASP hasta que se llegue al criterio de parada y devuelve el minimo peso
    cout.precision(4);
    iteracionActual_ = 0;
    ultimaIteracionConMejora_ = 0;
    while( ! parar(criterioParada) ) {
        iteracionActual_++;
        particionActual_ = h_.resolver();
        pesoParticionActual_ = pesoParticion(particionActual_);
        busquedaLocalUnNodo();
        if (pesoParticionActual_ < pesoMejorParticion_) {
            mejorParticion_ = particionActual_;
            pesoMejorParticion_ = pesoParticionActual_;
            ultimaIteracionConMejora_ = iteracionActual_;
        }
    }
    return pesoMejorParticion_;
}
\end{lstlisting}

\newpage

\subsection{Generador de instancias}

\lstset{language=C++,
        keywordstyle=\color{blue},
        stringstyle=\color{red},
        commentstyle=\color{magenta},
        morecomment=[l][\color{magenta}]{\#}
}
\begin{lstlisting}[frame=single]
typedef int Vertice;

int CANT_INSTANCIAS;
int MIN_VERTICES;
int MAX_VERTICES;
double MAX_COSTO_ARISTA;

Vertice seleccionarVerticeRandom(const set<Vertice> & conjunto) {
    int i = rand() % conjunto.size();
    auto it = conjunto.begin();
    for(int j = 0; j < i; j++) it++;
    return *it;
}

int main(int argc, const char* argv[]) {
    ifstream archivoConfiguracion("configuracionGeneracionInstancias.txt");
    archivoConfiguracion >> CANT_INSTANCIAS >> MIN_VERTICES >> MAX_VERTICES >> MAX_COSTO_ARISTA;
    archivoConfiguracion.close();
    srand(time(NULL) + getpid()); // Seedeo
    cout.precision(4);
    for (int n = MIN_VERTICES; n <= MAX_VERTICES; n++) {
        const int m_maximo = n * (n - 1) / 2; // Cantidad de aristas del grafo completo de n nodos
        const int m_minimo = 0.7 * m_maximo; // Minima cantidad de aristas de los grafos generados
        const int k_minimo = 2; // Minimo k de las instancias generadas
        const int k_maximo = max(k_minimo, n / 3); // Maximo k de las instancias generadas
        for (int i = 1; i <= CANT_INSTANCIAS; i++) {
            int m = m_minimo + rand() % (m_maximo - m_minimo + 1); // m es un valor aleatorio entre m_minimo y m_maximo
            int k = k_minimo + rand() % (k_maximo - k_minimo + 1); // k es un valor aleatorio entre k_minimo y k_maximo
            cout << n << " " << m << " " << k << endl;
            set<Vertice> vertices; // Vértices que todavía tienen al menos una arista disponible
            for (int i = 0; i < n; i++) {
                vertices.insert(i); // Los vertices van a ser enteros entre 0 y n-1, tengo que sumarles uno al imprimir
            }
            vector<set<Vertice>> vecinosPosibles(n);
            for (int i = 0; i < n; i++) {
                for (int j = 0; j < n; j++) {
                    if (i != j) {
                        vecinosPosibles[i].insert(j);
                    }
                }
            }
            while (m > 0) {
                Vertice v = seleccionarVerticeRandom(vertices);
                if (vecinosPosibles[v].size() == 0) {
                    vertices.erase(v);
                } else {
                    m--;
                    Vertice w = seleccionarVerticeRandom(vecinosPosibles[v]);
                    vecinosPosibles[v].erase(w);
                    vecinosPosibles[w].erase(v);
                    double weight = MAX_COSTO_ARISTA * (static_cast<double>(rand()) / static_cast<double>(RAND_MAX));
                    if (weight == 0.f) { // 0.f es equivalente a no tener arista, hay que arreglarlo
                        weight += 0.0001; // Esto sólo tiene sentido para cout.precision(r) con r >= 4
                    }
                    cout << v + 1 << " " << w + 1 << " " << fixed << weight << endl;
                }
            }
        }
    }
    return 0;
}
\end{lstlisting}
