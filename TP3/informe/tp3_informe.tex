\documentclass[11pt,a4paper]{article}
\usepackage{listings}
\usepackage{a4wide}
\usepackage{amsmath}
\usepackage{amssymb}
\usepackage{amsfonts}
\usepackage{verbatim}
\usepackage{fancyvrb}
\usepackage{fixltx2e}
\usepackage[utf8]{inputenc}
%\usepackage[latin1]{inputenc}
\usepackage[T1]{fontenc}
\usepackage[spanish]{babel}
\usepackage{indentfirst}
\usepackage{fancyhdr}
\usepackage{latexsym}
\usepackage{lastpage}
\usepackage{caratula}
\usepackage[colorlinks=true, linkcolor=black]{hyperref}
\usepackage{calc}
\usepackage{alltt}
\usepackage{verbatim}
\usepackage{caption}
\usepackage{hyperref}
\usepackage{url}
\usepackage{graphicx, subfigure}
\usepackage{float}
\usepackage{algorithm}
\usepackage{algorithmic}
%\input{spanishAlgorithmic}
\parskip = 11pt


\usepackage{enumerate}

% Pseudocódigo a lo Cormen
\usepackage{package/clrscode}

% auxiliares
\newcommand{\sisosi}{\Leftrightarrow}

\newcommand{\Frac}{\displaystyle\frac}

\newcommand{\suma}[2]{\sum\limits_{#1}^{#2}}

\newcommand{\Suma}[2]{\displaystyle\sum\limits_{#1}^{#2}}

\newcommand{\bc}{\begin{center}}
\newcommand{\ec}{\end{center}}


%\addtolength{\hoffset}{-1cm}
%\addtolength{\textwidth}{2cm}
 \addtolength{\voffset}{-0.5cm}
 \addtolength{\textheight}{1cm}
 
 %Datos de la caratula
\materia{Algorítmos y Estructura de Datos III}
\subtitulo{Heuristicas}
\titulo{Trabajo Práctico 3}

\fecha{19 / 12 / 2014}
\integrante{Ariel Paez}{668/09}{twizt.hl@gmail.com}
\integrante{Patito Campanna}{866/10}{patriciocapanna@gmail.com}
\integrante{Augusto Lamastica}{954/11}{mascittija@gmail.com}


\newcommand{\real}{\mathbb{R}}
\newcommand{\nat}{\mathbb{N}}
\newcommand{\eme}{\mathcal{M}}
\newcommand{\emeh}{\widehat{\mathcal{M}}}
\newcommand{\ere}{\mathcal{R}}

%%%%%%%%% MACROS %%%%%%%%%
\urldef\wikiMakefile\url{http://es.wikipedia.org/wiki/Make}
\urldef\mnistURL\url{http://yann.lecun.com/exdb/mnist/}
\urldef\wikiHouseholder\url{http://en.wikipedia.org/wiki/QR_decomposition#Using_Householder_reflections}

%%%%%%%% FIN MACROS %%%%%%

\parskip=5pt % 10pt es el tama�o de fuente

\begin{document}

\thispagestyle{empty}
\maketitle
\tableofcontents

\newpage
\section{k-Partici\'on de M\'inimo Peso (k-PMP)}
\subsection{Introducci\'on}
En este \'ultimo Trabajo Pr\'actico de la materia vamos a afrontar la resoluci\'on del problema de encontrar una k-Partici\'o de Peso M\'inimo (de ahora en m\'as k-PMP), el cual se trata de un problema que pertenece a la clase NP-Completo\footnote{http://en.wikipedia.org/wiki/Minimum\_k-cut} y por lo tanto no se conoce un algoritmo en tiempo polinomial que lo resuelva.\\
Por tal motivo, vamos a buscar una soluci\'on exacta mediante un BackTraking y luego realizar algoritmos aplicando distintas heur\'isticas (Golosa, Busqueda Local, GRASP) para analizar las ventajas y desventajas de cada una.

\subsection{Problema}

Dado un grafo simple $G = (V, E)$ y un entero $k$, se define una k-partici\'on de G como una partici\'on de V en k conjuntos de v\'ertices $V_{1} , \ldots , V_{k}$. Las aristas intrapartici\'on de una k-partici\'on, son aquellas aristas de G cuyos extremos se encuentran en un mismo conjunto de la partici\'on, es decir, una arista $uv \in E$ es intrapartici\'on si existe un $i \in {1, \ldots , k}$, tal que $u, v \in V_{i}$ . Dada una función de peso $\omega : E \rightarrow \mathbb{R}_+$, definida sobre las aristas de G, el peso de una k-partici\'on es la suma de los pesos de las aristas intrapartici\'on.

El problema $k$-PMP consiste en encontrar, dado un grafo simple $G = (V, E)$, un entero $k$ y una función $\omega : E \rightarrow \mathbb{R}_+$, una $k$-partición, que tenga a lo sumo $k$ subconjuntos, de $V$ con peso mínimo.

\subsection{Ejemplos}
En el seiguiente ejemplo vamos a ver lo que ser\'ia una partici\'on cualquiera del grafo y luego la k-PMP

Se el siguiente grafo: 

\bc
		\includegraphics[scale=0.5]{img/kpmp.png}
\ec

Podemos realizar la siguiente 2-partici\'on 

\bc
		\includegraphics[scale=0.5]{img/kpmp1.png}
\ec

Se puede apreciar f\'acilmente que esta partici\'on no cumple que sea m\'inima dentro de todas las 2-particiones posibles. Pues el peso total es de 54. Y por ejemplo existe la siguiente 2-particion en donde la suma del peso de sus aristas intraparticiones es igual a 0.

\bc
		\includegraphics[scale=0.5]{img/kpmp2.png}
\ec

\newpage
\section{Comparativas $k$-PMP}
\subsection{Descripci\'on del problema}

Nos encontramos ante una competencia que consiste en cruzar un puente dando saltos con la menor cantidad posible de estos.
Cada participante tiene una capacidad de salto o salto máximo limitado, razón por la cual no pueden cruzar el puente de un salto, a menos que su salto sea mayor a la longitud del puente. Esta longitud la medimos por la cantidad de tablones que posee el puente.
Para no hacer demasiado trivial la competencia, no todos los tablones del puente están en buen estado, algunos están rotos, pero están marcados para que no se salte a estos. Esto trae una consecuencia, puede darse el caso que el salto máximo de un participante sea n y en el puente haya m tablones rotos continuos, con m > n, por lo tanto cuando el participante llegue al tablón anterior a esta serie continua de tablones rotos, no podrá efectuar mas saltos, y pierde el juego. 

\subsection{Resoluci\'on}


\subsection{Demostraci\'on de la resoluci\'on}

\subsection{Complejidad del algoritmo}


\subsection{C\'odigo fuente}

\lstset{language=C++,
                basicstyle=\ttfamily\footnotesize,
                keywordstyle=\color{blue}\ttfamily,
                stringstyle=\color{red}\ttfamily,
                commentstyle=\color{green}\ttfamily,
                morecomment=[l][\color{magenta}]{\#},
                breaklines=true
}
\begin{lstlisting}

typedef std::vector<int> LTablonesEstado;
typedef std::vector<int> LSaltos;

LSaltos resolver(int cantTablones, int saltoMaximo, LTablonesEstado& tablones){

	LSaltos saltos;
	int contadorSaltoMaximo = 0;
	int tablonASaltar;
	int tamSalto = 0;
	bool encontroSalto = false;

	for(int nroTablon = 0; nroTablon < cantTablones; nroTablon++){

		contadorSaltoMaximo++;
		tamSalto++;
		
		//Si hay un tablon en ese lugar actualizo tablonASaltar
		if(tablones[nroTablon] == 1){
			tablonASaltar = nroTablon;
			encontroSalto = true;
		}
		
		//Si llegue al limite de salto agrego (Y pude hacer un salto) agrego el tablon al que voy a saltar a la lista de saltos y reinicio los contadores
		if(contadorSaltoMaximo == saltoMaximo && encontroSalto){			
			//Guardo el tablon al que salte
			saltos.push_back(tablonASaltar);

			//reinicio las variables para un nuevo salto
			contadorSaltoMaximo = saltoMaximo - tamSalto;
			tamSalto = 0;
			encontroSalto = false;

		}else if (contadorSaltoMaximo == saltoMaximo && !encontroSalto)
		{
			saltos.clear();
			break;
		}
	}
	return saltos;//Tambien devolver si se pudo llegar al final con saltos.
}

\end{lstlisting}

\subsection{Casos de prueba}

Para este ejercicio escogimos algunos casos de prueba que tienen las siguientes características:

\begin{itemize}

\item Caso en el que el puente no posee ningún tablón roto. Aquí el participante debería cruzar el puente dando saltos de tamaño igual a su salto máximo, ya que no hay nada que impida que no lo haga.

\item Caso en el que el puente no posee tablones sanos, y el limite de salto del jugador es menor a la cantidad de tablones del puente. El participante no podrá hacer ningún salto, ya que no importa de que longitud sea este, siempre terminará cayendo en un tablón roto.

\item Caso en el que el puente no posee tablones sanos, y el limite de salto del jugador es mayor a la cantidad de tablones del puente. En este caso el participante podrá cruzar el puente de un salto igual al tamaño del puente mas uno.

\item Caso en el que los tablones sanos están cada $x$ tablones, donde $x$ es igual al salto máximo del participante. En este caso el participante cruzara el puente en $m$ cantidad de saltos, donde $m$ es la cantidad de tablones sanos. Osea que pasará por todos los tablones sanos.

\item Caso en el cual los últimos $m$ tablones están rotos, por lo tanto el algoritmo debe ejecutarse con normalidad, pero a la hora de evaluar el ultimo salto, este no podrá efectuarse y se debe borrar la lista de saltos hechos y retornar "no".

\end{itemize}


\subsection{Performance}


\newpage
\section{Algoritmo exacto: Backtraking}
\newpage
\section{Problema 2}

\input{introduccion2.tex}


\newpage
\section{Heur\'istica Golosa}
\subsection{Resoluci\'on}

Para nuestra heurística golosa comenzamos ordenando las aristas por peso de mayor a menor. Una vez obtenida la lista de aristas ordenadas por peso la iteramos escogiendo primero un nodo (arbitrariamente) de la arista más pesada y si la arista aun no se encuentra ubicada, se agrega a un nuevo conjunto siempre y cuando no hayamos sobrepasado la cantidad $k$ de conjuntos creados. En caso de que ya tengamos $k$ conjuntos o en caso de que hayamos terminado de recorrer las aristas se termina el ciclo.
Ahora verificamos si la cantidad de conjuntos creados es menor a $k$, en caso afirmativo habremos ubicado todos los nodos con aristas en alguno de los conjuntos.
Si quedan nodos sin ubicar, estos se agregan a cualquier conjunto, como ya agregamos todos los nodos de grado mayor a uno, los que restan se puede decir que tienen grado 0, por lo tanto, no importa en que conjunto sean agregados, estos no crearán aristas intraparticion y como consecuencia no sumarán peso a ningún conjunto.
Si la cantidad de conjuntos creados es igual a $k$, actuamos de forma diferente, aquí recorremos todas las aristas, actuando solo con las que no hayan sido agregadas a ningún conjunto de la siguiente manera: verificamos cuanto peso agregaría en cada conjunto para luego realmente añadirlo al conjunto que sume menos peso, en caso de encontrar uno en el que no cree aristas intraparticion, es agregado a este sin seguir revisando los restantes.

Como se puede observar usamos dos componentes greedys en la heurística:
La primera esta cuando se ordenan las aristas decrecientemente y se crean los $k$ conjuntos a medida que se separan los nodos de las aristas de mayor peso, para que estos no generen aristas intraparticion.
La segunda componente greedy se encuentra cuando al finalizar de crear los $k$ conjuntos se recorre nodo por nodo verificando si ya se encuentra agregado y en caso negativo, verificando en que conjunto genera menos peso para finalmente agregarlo a este.

A modo de ejemplo presentamos algunas imágenes para mostrar el funcionamiento de la heurística:

\begin{figure}[H]
\begin{center}
\includegraphics[scale=0.4]{./img/greedy1.png}
\caption{(1) Grafo de ejemplo}
\end{center}
\end{figure}
La primer imagen corresponde al grafo al que se le aplicará k-PMP (1)

\begin{figure}[H]
\begin{center}
\includegraphics[scale=0.4]{./img/greedy2.png}
\caption{(2) Conjuntos 1 y 2 luego de separar los nodos de las aristas mas pesadas}
\end{center}
\end{figure}
Luego de ordenar las aristas escoge las de mayor tamaño y separa sus nodos en k conjuntos k=2 para este ejemplo. (2)

\begin{figure}[H]
\begin{center}
\includegraphics[scale=0.4]{./img/greedy3.png}
\caption{(3) Conjuntos luego de comenzar a ingresar los nodos restantes}
\end{center}
\end{figure}
Luego como ya hay 2 conjuntos creados procede a verificar nodo los nodos y agregándolos al conjunto que menos sume. En el caso del nodo 1, se agrega de inmediato al primer conjunto porque no genera arista intraparticion. Para el nodo 2 primero verifica en el 1er conjunto, aquí sumaria 2 de peso, se verifica en el siguiente conjunto, y como no suma peso, se ingresa ahí. (3)

\begin{figure}[H]
\begin{center}
\includegraphics[scale=0.4]{./img/greedy4.png}
\caption{(4) Resultado de haber corrido la heurística greedy al grafo (1)}
\end{center}
\end{figure}
Finalmente resta agregar el nodo 3 el cual genera en el conjunto uno un peso igual a 5, y en el conjunto 2 un peso igual a 7 al unirse con los nodos 2 y 4. Por lo tanto el nodo 3 es introducido en el conjunto 1. (4)


A continuación presentamos un pseudo-gráfico del algoritmo
\begin{itemize}
\item ordenar $aristas$ por peso en orden decreciente
\item mientras restan $aristas$ 
  \begin{itemize}
  \item si no hay $k$ conjuntos, crear conjunto y agregar $nodo1$ de $arista$ (si este no pertenece a ningún conjunto)
  \item sino cortar el ciclo  
  \item si no hay $k$ conjuntos, crear conjunto y agregar $nodo2$ de $arista$ (si este no pertenece a ningún conjunto)
  \item sino cortar el ciclo  
  \end{itemize}
\item Si hay menos de $k$ conjuntos creados
  \begin{itemize}
  \item agregar las aristas que quedaron fuera a alguno de los conjuntos creados
  \end{itemize}
\item Sino
  \begin{itemize}
  \item desde nodo 0 a nodo n-1
    \begin{itemize}
    \item si el nodo no esta en ningún conjunto
      \begin{itemize}
        \item verificar conjunto por conjunto cuanto peso generaría agregarlo a este
        \item agregar el nodo al conjunto en el que genere menos peso
      \end{itemize}
    \end{itemize}
  \end{itemize}
\end{itemize}




\subsection{Análisis de complejidad}

Vamos a analizar el código paso por paso analizando la complejidad temporal del peor caso.
Lo primero que hacemos es ordenar las aristas, para esto utilizamos la función $sort$ de la $std$ que posee una complejidad temporal de $O(n.log(n))$, en este caso como se aplica a las aristas esto es $O(m.log(m))$ siendo $m$ la cantidad de aristas.
A continuación, y haciendo uso del pseudocódigo proporcionado:
\begin{itemize}
\item mientras restan $aristas$ 
  \begin{itemize}
  \item si no hay $k$ conjuntos, crear conjunto y agregar $nodo1$ de $arista$ (si este no pertenece a ningún conjunto)
  \item sino cortar el ciclo  
  \item si no hay $k$ conjuntos, crear conjunto y agregar $nodo2$ de $arista$ (si este no pertenece a ningún conjunto)
  \item sino cortar el ciclo  
  \end{itemize}
\end{itemize}
Este scope tiene una complejidad $O(m)$ ya que se recorren todas las aristas, en el peor caso creando conjuntos y agregándolos a estos, pero estas dos acciones tienen una complejidad constante.

A continuación si se crearon menos de $k$ conjuntos se agregan los nodos que restan a uno de estos. En el peor caso esto pertenece al orden de $O(n)$ siendo n la cantidad de nodos.
Si hay $k$ conjuntos creados se procede de la siguiente manera:
\begin{itemize}
\item desde nodo 0 a nodo n-1   $O(n)$
  \begin{itemize}
  \item si el nodo no esta en ningún conjunto $O(1)$
    \begin{itemize}
      \item verificar conjunto por conjunto cuanto peso generaría agregarlo a este $O(k+n)$
      \item agregar el nodo al conjunto en el que genere menos peso $O(1)$
    \end{itemize}
  \end{itemize}
\end{itemize}

Esta ultima parte del algoritmo tiene una complejidad acotada en el peor caso de O(n^2) ya que por cada nodo no agregado ($n$ en total en el peor caso) se verifica cuanto pesaría agregarlo a cada conjunto. Para esto se recorren los nodos ya ingresados en el conjunto actual (peor caso $n$) y se agrega al que menor peso aporte. Esto tiene relación con el $k$ también ya que en el peor caso se verificará en los $k$ conjuntos.

\subsection{Instancias desfavorables}


Probamos la heurística con varias familias de grafos, grafos estrella...(ETC) 
Para el caso de los grafos estrella nuestra heurística funcionaba mal, particularmente cuando el numero del nodo central era superior a la cantidad k de conjuntos. 
Para aclarar, en un principio nuestro algoritmo constaba solo de la segunda componente greedy de la que se habla en la sección Resolución. Por lo que comenzaba desde el primer nodo hasta el ultimo probando en que conjunto convenía poner dicho nodo de forma de disminuir lo mas posible el peso total de las aristas intraparticion. Volviendo al caso del grafo estrella, lo que pasaba era que ingresaba en los

\subsection{Experimentación}

Para poder observar la performance del algoritmo en términos de tiempo de ejecución en función al tamaño de la entrada nos construimos un script en python que genera tres tipos de grafos, grafos con un 15 \% de aristas, uno con un 50 \% de aristas y el último con el 100 \% de aristas, tomando para el porcentaje la cantidad de aristas que llevaría un grafo completo.
Para cada uno de estos variamos los valores de n y de k con n entre 100 y 500. 
Por cada uno de esos valores de n, variamos el valor de k entre 2 y 10.
Cada una de todas estas instancias fueron corridas entre 20 y 30 veces, (20 para las instancias mas grandes) para poder calcular un promedio. Ya que consideramos que el procesador no esta ejecutando solo nuestra heurística, por lo que el tiempo que le toma correr el algoritmo hasta obtener una solución seguramente es mayor al real. Por otro lado también pensamos en la posibilidad de que el procesador cachee las soluciones al estar haciendo muchas veces lo mismo, razón por la cual decidimos no superar las 30 repeticiones. Entre estas dos cosas donde una desfavorece y otra favorece el tiempo de ejecución, nos pareció razonable realizar un promedio.
Con los valores obtenidos presentamos algunos gráficos para tener una descripción mas visual y los analizamos.

\begin{figure}[H]
\begin{center}
\includegraphics[scale=0.4]{./img/greedyN1.png}
\caption{Gráficos con k = 2}
\end{center}
\end{figure}

\begin{figure}[H]
\begin{center}
\includegraphics[scale=0.4]{./img/greedyN2.png}
\caption{Gráficos con k = 5}
\end{center}
\end{figure}

\begin{figure}[H]
\begin{center}
\includegraphics[scale=0.4]{./img/greedyN3.png}
\caption{Gráficos con k = 10}
\end{center}
\end{figure}


En estos gráficos podemos observar que el algoritmo tiene una complejidad de n^2 en relación a la cantidad de nodos, la cual obviamente se ve afectada al mismo tiempo por la cantidad de aristas que el grafo posee. Esta observación se puede notar ya que cada grafo presenta la complejidad temporal para una densidad de aristas de 15\%, 50\% y 100\% comparada con un valor n^2 multiplicado por una constante obtenida a partir de dividir el tiempo sobre la cantidad de nodos, y a su vez este valor es multiplicado por 0.15, 0.50 y 1.
Al hacer esto y verificar que los valores se asemejan demasiado podemos concluir que la cantidad de aristas del grafo afecta directamente al rendimiento de forma lineal.

En los tres gráficos utilizamos los mismos valores para O(n^2)* densidad de cantidad de aristas, y como se puede observar comparando los distintos gráficos, que presentan una variación en la cantidad k de conjuntos, los valores resultantes de los tests son casi los mismos, no se nota ninguna variación apreciable. 



%\begin{figure}[H]
%\begin{center}
%\includegraphics[scale=0.4]{./imágenes/ej1_chartRendimiento.png}
%\caption{Gr\'afico de tiempos con test pseudo-aleatorios.}
%\end{center}
%end{figure}



\newpage
\section{Heur\'istica de B\'usqueda Local}
\newpage
\section{Problema 4}

\input{introdiccion4.tex}


\newpage
\section{Heur\'istica GRASP}
\subsection{Resoluci\'on}

La metaheuristica GRASP es una combinacion de una heuristica golosa “aleatorizada” y un procedimiento de busqueda local. En detalle GRASP se define de la siguiente manera:

\begin{itemize}
\item Mientras no se alcance el \textbf{criterio de terminacion}
  \begin{itemize}
  \item Obtener $s \in S$ mediante una heuristica golosa \textbf{aleatorizada}.
  \item Mejorar $s$ mediante busqueda local
  \item Recordar la mejor solucion obtenida hasta el momento.
  \end{itemize}
\end{itemize}

Para nuestro GRASP, basamos la heuristica golosa aleatorizada en la segunda componente golosa descrita en el punto 4. \\
Entrando en detalle, la heuristica golosa itera sobre todos los vertices y decide sobre cada uno a que particion va a ser agregada. La manera en que lo hacia en la heuristica golosa previamente especificada era elegir la particion tal que el peso agregado a la solucion sea el menor posible. Para ello requeria iterar sobre todos las Particiones y preguntarse cuanto pesaria de ser agregada alli. \\
En la heuristica randomizada cambia en que primero comenzamos con k particiones vacias. Y segundo que en vez de elegir directamente la mejor opcion, armamos una RCL (lista Restringida de Candidatos), donde los candidatos son Particiones. Luego se elige aleatoriamente un destino cualquiera entre los candidatos y se agrega el vertice en el.\\
Definimos la restriccion de los candidatos utilizando dos parametros alpha y beta. \\
Decimos que los candidatos no pueden tener un valor menor que un cierto porcentaje alpha del valor del mejor candidato. Empezamos por identificar al mejor candidato como el menor peso agregado a la solucion, y al peor candidato como el mayor peso agregado a la solucion. La restriccion luego excluye a aquellos candidatos cuyo peso agregado a la solucion no se comprenda entre el menor peso y el mayor peso menos el porcentaje alpha de la distancia entre ambos pesos.\\
Por ultimo tambien la RCL puede contener como maximo los beta mejores candidatos.

En cuanto al procedimiento de busqueda local utilizamos nada mas y nada menos que la misma heuristica concebida en el punto anterior.\\

Y finalmente como criterios de terminacion, definimos dos. Uno es la cantidad maxima de iteraciones. Es decir la cantidad de veces que se itera sobre el ciclo principal de GRASP. Y el otro la cantidad maxima de iteraciones seguidas sin cambios. Es decir el maximo de veces que itera sobre el ciclo principal, luego de haber encontrado una mejora, sin encontrar otra.\\

A continuación presentamos un pseudo-codigo del algoritmo goloso aleatorizado:\\
\begin{itemize}
\item desde vertice 1 a vertice n
    \begin{itemize}
    \item por cada particion existente
        \begin{itemize}
        \item verificamos el peso de agregar el vertice a la particion
        \item Si el peso resultado de agregarlo es el maximo encontrado hasta el momento lo recuerdo
        \item Si el peso resultado de agregarlo es el minimo encontrado hasta el momento lo recuerdo
        \item Agregamos la particion destino con peso asociado a la lista RLC
        \item Si el tamano de RLC es mayor a beta:
            \begin{itemize}
            \item Eliminamos la particion destino con mayor peso
            \end{itemize}
        \end{itemize}
    \end{itemize}
    \item Eliminamos de RLC las particion destino cuyo valor de peso no se encuentre entre el rango alpha de mejores
    \item Elegimos aleatoriamente una particion destino entre las candidatas y agregamos el vertice en esa particion 
\end{itemize}


\subsection{Analisis de complejidad}

Analisamos la complejidad del pseudo-codigo del algoritmo goloso aleatorizado:

Lo primero que hacemos es iterar por todos los vertices. De 1 a n. Esto es O(n)
\begin{itemize}
\item desde vertice 1 a vertice n       $O(n)$
    \begin{itemize}
    \item por cada particion existente    $O(k)$
        \begin{itemize}
        \item verificamos el peso de agregar el vertice a la particion     $O(n?)$
        \item Si el peso resultado de agregarlo es el maximo encontrado hasta el momento lo recuerdo $O(1)$
        \item Si el peso resultado de agregarlo es el minimo encontrado hasta el momento lo recuerdo $O(1)$
        \item Agregamos la particion destino con peso asociado a la lista RLC (ORDENADAMENTE)  $O(min(k,beta))$
        \item Si el tamano de RLC es mayor a beta: $O(1)$
            \begin{itemize}
            \item Eliminamos la particion destino con mayor peso  $O(1)$
            \end{itemize}
        \end{itemize}
    \end{itemize}
    \item Eliminamos de RLC las particion destino cuyo valor de peso no se encuentre entre el rango alpha de mejores $O(min(beta, k))$
    \item Elegimos aleatoriamente una particion destino entre las candidatas y agregamos el vertice en esa particion $O(min(beta, k))$
\end{itemize}

Haciendo un analisis mas profundo podemos notar que el costo en complejidad de verificar el peso de agregar el vertice a una particion es: O(1) por la $k_{i}$ particion + la cantidad de vertices que contenga. El costo para un vertice i luego es O(i-1) + O(k) comparaciones por todas las particiones y los vertices totales contenido en ellas, i es creciente desde 1 hasta n. La complejidad resultado acotando a k por n es O(n). \\
Agregar ordenadamente a una lista ordenada cuesta el tamano de la lista. Y la lista contiene como maximo todas las particiones, es decir como maximo $k$. Recordemos que $k$ podemos acotarlo por $n$, entonces la complejidad es absorida por el item anterior deducido O(n).
La complejidad de eliminar particiones de una lista y de elegir aleatoriamente tambien se acotan por O(n). La complejidad final es O(n)*O(n) = O(n2). \\

La complejidad de busqueda local ya la calculamos en el punto anterior y es O(n2).

Finalmente la complejidad de Grasp es O(maxIteraciones*n2). Dado que logicamente la cantidad de iteraciones seguidas sin cambios es menor a la cota maxima cantidad de iteraciones.

Los costos de copia de lista de particiones como resultado de busqueda local o greedy random o comparacion con anterior mejor solucion son despreciados por encontrar un maximo costo de O(n).


\subsection{Criterios de parada y RCL}




\newpage
\section{Conclusiones}
En esta seccion vamos a realizar una conclusi\'on general de los distintos algoritmos. En base a la performance de los mismos y las distancias con respecto a la soluci\'on Exacta.

Primero planteamos un experimento en el cual corrimos para instancias aleatoreas, nuevamente con las tres densidades utilizadas a lo largo de todo el Trabajo Pr\'actico(15\%, 50\% y 100\%), todos los algoritmos (Exacto con podas, Heur\'istica Golosa, Heur\'istica de Busqueda Local y Heur\'istica GRASP con 5, 15 y 40 iteracioens respectivamente) unas 50 veces para cada cantidad de nodos entre 5 y 20 (por ser numeros manejables para el algoritmo exacto) para sacar un promedio de cuanto difiere cada Heur\'istica contra el algoritmo Exacto.\\
Para esto calculamos la peor soluci\'on para cada instancia (que ser\'ia ubicar todos los nodos en una \'unica partici\'on, con lo cual la sumatoria de las aristas intrapartici\'on ser\'a la sumatoria total de las aristas) y generamos una formula que nos representa un valor entre 1 y 0 que tanto diverge la Heur\'istica en relaci\'on al exacto, pero considerando el valor entre la peor soluci\'on y la soluci\'on ideal.
La formula es: 

\bc
1 - ((solucionHeur\'istica - solucionExacto)/(peorSolucion - solucionExacto))
\ec

Si la Heur\'istica tiene el mismo valor que el Exacto esta relaci\'on nos devuelve que es 1. Todo lo contrario si la heur\'istica devuelve la peor soluci\'on, esta formula nos retorna 0.\\
Con lo cual es un buen indicador para saber cuan buena es la soluci\'on de una de las heur\'isticas.

\subsection{Divergencia de las Heur\'isticas}

\subsubsection{2 Particiones}

\begin{figure}[H]
\begin{center}
\includegraphics[scale=0.3]{finales/ComparacionesCon2Particiones15Aristas.png}
\caption{Distancias de las Heur\'isticas para K = 2 y 15\% de aristas}
\end{center}
\end{figure}

\begin{figure}[H]
\begin{center}
\includegraphics[scale=0.3]{finales/ComparacionesCon2Particiones50Aristas.png}
\caption{Distancias de las Heur\'isticas para K = 2 y 50\% de aristas}
\end{center}
\end{figure}

\begin{figure}[H]
\begin{center}
\includegraphics[scale=0.3]{finales/ComparacionesCon2Particiones100Aristas.png}
\caption{Distancias de las Heur\'isticas para K = 2 y 100\% de aristas}
\end{center}
\end{figure}

\subsubsection{3 Particiones}

\begin{figure}[H]
\begin{center}
\includegraphics[scale=0.3]{finales/ComparacionesCon3Particiones15Aristas.png}
\caption{Distancias de las Heur\'isticas para K = 3 y 15\% de aristas}
\end{center}
\end{figure}

\begin{figure}[H]
\begin{center}
\includegraphics[scale=0.3]{finales/ComparacionesCon3Particiones50Aristas.png}
\caption{Distancias de las Heur\'isticas para K = 3 y 50\% de aristas}
\end{center}
\end{figure}

\begin{figure}[H]
\begin{center}
\includegraphics[scale=0.3]{finales/ComparacionesCon3Particiones100Aristas.png}
\caption{Distancias de las Heur\'isticas para K = 3 y 100\% de aristas}
\end{center}
\end{figure}


\subsubsection{4 Particiones}

\begin{figure}[H]
\begin{center}
\includegraphics[scale=0.3]{finales/ComparacionesCon4Particiones15Aristas.png}
\caption{Distancias de las Heur\'isticas para K = 4 y 15\% de aristas}
\end{center}
\end{figure}

\begin{figure}[H]
\begin{center}
\includegraphics[scale=0.3]{finales/ComparacionesCon4Particiones50Aristas.png}
\caption{Distancias de las Heur\'isticas para K = 4 y 50\% de aristas}
\end{center}
\end{figure}

\begin{figure}[H]
\begin{center}
\includegraphics[scale=0.3]{finales/ComparacionesCon4Particiones100Aristas.png}
\caption{Distancias de las Heur\'isticas para K = 4 y 100\% de aristas}
\end{center}
\end{figure}

\subsubsection{5 Particiones}

\begin{figure}[H]
\begin{center}
\includegraphics[scale=0.3]{finales/ComparacionesCon5Particiones15Aristas.png}
\caption{Distancias de las Heur\'isticas para K = 5 y 15\% de aristas}
\end{center}
\end{figure}

\begin{figure}[H]
\begin{center}
\includegraphics[scale=0.3]{finales/ComparacionesCon5Particiones50Aristas.png}
\caption{Distancias de las Heur\'isticas para K = 5 y 50\% de aristas}
\end{center}
\end{figure}

\begin{figure}[H]
\begin{center}
\includegraphics[scale=0.3]{finales/ComparacionesCon5Particiones100Aristas.png}
\caption{Distancias de las Heur\'isticas para K = 5 y 100\% de aristas}
\end{center}
\end{figure}

\subsubsection{6 Particiones}

\begin{figure}[H]
\begin{center}
\includegraphics[scale=0.3]{finales/ComparacionesCon6Particiones15Aristas.png}
\caption{Distancias de las Heur\'isticas para K = 6 y 15\% de aristas}
\end{center}
\end{figure}

\begin{figure}[H]
\begin{center}
\includegraphics[scale=0.3]{finales/ComparacionesCon6Particiones50Aristas.png}
\caption{Distancias de las Heur\'isticas para K = 6 y 50\% de aristas}
\end{center}
\end{figure}

\begin{figure}[H]
\begin{center}
\includegraphics[scale=0.3]{finales/ComparacionesCon6Particiones100Aristas.png}
\caption{Distancias de las Heur\'isticas para K = 6 y 100\% de aristas}
\end{center}
\end{figure}

\subsubsection{7 Particiones}

\begin{figure}[H]
\begin{center}
\includegraphics[scale=0.3]{finales/ComparacionesCon7Particiones15Aristas.png}
\caption{Distancias de las Heur\'isticas para K = 7 y 15\% de aristas}
\end{center}
\end{figure}

\begin{figure}[H]
\begin{center}
\includegraphics[scale=0.3]{finales/ComparacionesCon7Particiones50Aristas.png}
\caption{Distancias de las Heur\'isticas para K = 7 y 50\% de aristas}
\end{center}
\end{figure}

\begin{figure}[H]
\begin{center}
\includegraphics[scale=0.3]{finales/ComparacionesCon7Particiones100Aristas.png}
\caption{Distancias de las Heur\'isticas para K = 7 y 100\% de aristas}
\end{center}
\end{figure}



Podemos ver como a medida que aumenta la cantida de nodos los valores de divergencia de las Heur\'isticas comienzan a aumentar.\\

Otro punto f\'acilmente apreciable es que la Heur\'istica Greedy es la que mayor distancia tiene con respecto a la soluci\'on Exacta para todos los valores de n y cantidad de particiones, seguida por Busqueda Local y luego, como es de esperarse, los GRASP con distinta cantidad de iteraciones.\\

Volvemos a ver como al aumentar la cantidad de particiones, en funcion de las densidades del grafo, para grafos menos densos, en todos los algoritmos las soluciones se acercan mucho mas a la Ecxacta. Como ya lo mencionamos en alg\'un momento esto debe ser produco de la libertad que tengo de colocar los nodos al tener un grafo poco denso (que probablemnte tanga una gran cantidad de nodos aislados y una menor aristas intrapartici\'ones).

No podemos apreciar bien los valores de los distintos GRASP con sus respectiva cantidad de repeticiones, ya que para estas cantidades de nodos y particiones todos devuelven resultados muy similares.\\
Por este motivo vamos a realizar m\'as experimentos pero excuptuando al algoritmo exacto para poder correr instancias m\'as grandes.


\section{Divergencia para instancias mayores}

Usamos el mismo cr\'iterio de comparacion que en el ejercicio anterior. Pero en vez de utilizar el excato como patr\'on utilizamos el GRASP con 40 repeticiones. Los valores para la cantidad de particiones seleccionados fueron 5, 30 y 70.

\subsubsection{5 Particiones}

\begin{figure}[H]
\begin{center}
\includegraphics[scale=0.4]{finales/muchosComparacionesCon5Particiones15Aristas.png}
\caption{Distancias de las soluciones para K = 5 y 15\% de aristas}
\end{center}
\end{figure}

\begin{figure}[H]
\begin{center}
\includegraphics[scale=0.4]{finales/muchosComparacionesCon5Particiones50Aristas.png}
\caption{Distancias de las solucioens para K = 5 y 50\% de aristas}
\end{center}
\end{figure}

\begin{figure}[H]
\begin{center}
\includegraphics[scale=0.4]{finales/muchosComparacionesCon5Particiones100Aristas.png}
\caption{Distancias de las solucioens para K = 5 y 100\% de aristas}
\end{center}
\end{figure}

\subsubsection{30 Particiones}

\begin{figure}[H]
\begin{center}
\includegraphics[scale=0.4]{finales/muchosComparacionesCon30Particiones15Aristas.png}
\caption{Distancias de las soluciones para K = 30 y 15\% de aristas}
\end{center}
\end{figure}

\begin{figure}[H]
\begin{center}
\includegraphics[scale=0.4]{finales/muchosComparacionesCon30Particiones50Aristas.png}
\caption{Distancias de las solucioens para K = 30 y 50\% de aristas}
\end{center}
\end{figure}

\begin{figure}[H]
\begin{center}
\includegraphics[scale=0.4]{finales/muchosComparacionesCon30Particiones100Aristas.png}
\caption{Distancias de las solucioens para K = 30 y 100\% de aristas}
\end{center}
\end{figure}

\subsubsection{70 Particiones}

\begin{figure}[H]
\begin{center}
\includegraphics[scale=0.4]{finales/muchosComparacionesCon70Particiones15Aristas.png}
\caption{Distancias de las soluciones para K = 70 y 15\% de aristas}
\end{center}
\end{figure}

\begin{figure}[H]
\begin{center}
\includegraphics[scale=0.4]{finales/muchosComparacionesCon70Particiones50Aristas.png}
\caption{Distancias de las solucioens para K = 70 y 50\% de aristas}
\end{center}
\end{figure}

\begin{figure}[H]
\begin{center}
\includegraphics[scale=0.4]{finales/muchosComparacionesCon70Particiones100Aristas.png}
\caption{Distancias de las solucioens para K = 70 y 100\% de aristas}
\end{center}
\end{figure}


Luego de estos neuvos experimentos tenemos nuevas conclusiones pero tamb\'ien nuevas dudas. Para empezar hay valores mayores a 1 para algunas corridas de los GRASP, que no tienen sentido, pues esto debendr\'ia de valores negativos en las soluciones de las Heur\'isticas, lo cual no puede presentarse nunca, por ser los peso de las aristas todos positivos. Son casos algunos pocos casos aislados con lo cual no afectan mucho nuestros experimentos.

Por otro lado, se puede corroborar claramente el orden de eficiencia de las Heur\'isticas, desde la peor siendo la Greedy, luego las Busqueda local, para finalizar con la GRASP con sus distinta cantidad de iteraciones (A mayor cantida de iteraciones, mayor calidad de la soluci\'on).\\

Luego tambien podemos ver como la cantidad de particiones y la densidad de los grafos marcan un corte desde el cual comienzan a incrementar las diferencias entre las soluciones de los distintos algoritmos. Por ejemplo vemo como para 30 particiones cuando la cantidad de nodos no supera los 30 todos los algoritmos retornan la soluci\'on optima que es la de peso 0. M\'as a\'un, para una densidad menor del grafo, con 30 particiones vemos como para el 50\% de las aristas siguen retornando todos la mejor soluci\'on (que en este caso no se puede afirmar que sea de peso 0) hasta casi instancias del problema con 90 nodos.

Luego de todo esto pudimos percibir ciertos comportamientos llamativos.

Por ejemplo hab\'iamos pensado que aumentando la cantidad de nodos y que la relaci\'on entre la cantida de particiones y cantidad de nodos, sea amplimente superada, iba a producir que la distancia de la soluci\'on de las Heur\'isticas se alejen cada vez m\'as de los valores \'optimos.

Pero por un lado podemos ver que para todos los algoritmos y sin importar la cantidad de nodos ni la relaci\'on entre las particiones y la cantidad de nodos, que la variaci\'on entre los mejores resultados y los peores es muy peque\'na. No llega a alejarse m\'as del 0.92 dentro de esta escala que fijamos para comparar resultados.\\

Lo m\'as llamativo es que todos los algoritmos parecieran a converger a un valor estable de distancia en relaci\'on a la mejor soluci\'on, como podemos ver en el gr\'afico para 5 particiones con el 100\% de las aristas.

\newpage
\begin{thebibliography}{2}

%\bibitem{sort}
%  C++ reference,
%  \url{http://www.cplusplus.com/reference/algorithm/sort/}
 

%\bibitem{upper}
%	C++ reference,
%	\url{http://www.cplusplus.com/reference/map/multimap/}
  
\end{thebibliography}



\end{document}
