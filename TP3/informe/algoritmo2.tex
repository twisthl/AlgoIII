Para resolver el problema del k-PMP vamos a utilizar un algoritmo que se basa en la t\'ecnica de Backtracking. Este recorre todas las posibles particiones del Grafo y basta con devolver aquella que cumpla que la sumatoria de los pesos de sus aristas intraparticiones sea m\'inimo y tenga menos de k-Particiones.
Luego se aplican algunas podas para mejorar un poco el tiempo de ejecuci\'on de esta soluci\'on, como guaradar la mejor soluci\'on hasta el momento o no calcular aquellas que tenga mas de k particiones pues no pueden ser soluci\'on del problema.

METER PSEUDO!!!!

\begin{codebox}
  \Procname{$\proc{Hash-Insert}(T,k)$}
  \li $i \gets 0$
  \li \Repeat
  \li   $j \gets h(k,i)$
  \li   \If $T[j] == \const{nil}$
  \li   \Then
          $T[j] \gets k$
  \li     \Return $j$
  \li   \Else
          $i \gets i+1$
        \End
        \For
        \While
  \li \Until $i == m$
  \li \Error ``hash table overflow''
\end{codebox}

\subsubsection{Podas}
La poda que aplicamos al algoritmo de BackTraking es bastante sencilla pero efectiva.\\
B\'asicamente me guardo la \'ultima mejor soluci\'on. Y si en el camino de recorrer las particiones, al agregar un nodo en alguna de las particiones, este genera una peor soluci\'on que la que ya tengo calculada, directamente la descarto y paso a intentar colocar este nodo en la proxima partici\'on.\\

En la secci\'on de experimentaci\'on vamos a mostrar los resultados de correr los algoritmos con o sin poda.




