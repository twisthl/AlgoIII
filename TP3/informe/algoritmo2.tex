Para resolver el problema del k-PMP vamos a utilizar un algoritmo que se basa en la t\'ecnica de Backtracking. Este recorre todas las posibles particiones del Grafo y basta con devolver aquella que cumpla que la sumatoria de los pesos de sus aristas intraparticiones sea m\'inimo y tenga menos de k-Particiones.
Luego se aplican algunas podas para mejorar un poco el tiempo de ejecuci\'on de esta soluci\'on, como guardar la mejor soluci\'on hasta el momento o no calcular aquellas que tenga mas de k particiones pues no pueden ser soluci\'on del problema.

Básicamente el algoritmo exacto va iterando los nodos y colocandolos en las particiones para ir verificando si voy generando una solución mejor. En el algoritmo sin podas agrego un nodo a una partición y calculo la sumatoria de las aristas intrapartición del grafo que si es la mínima hasta el momento, la guardo como solución del problema, para luego llamar recursivamente a la función con el próximo nodo. Luego de salir de la recursión saco el nodo que habia colocado en la particion y pruebo colocandolo en la próxima partición para reiterar los cáluclos y ver si puedo disminuir el peso de la sumatoria de las aristas intrapartición.

\begin{codebox}
  \Procname{$\proc{combinar}(list<Particion> particiones, Vertice verticeAUbicar, double pesoAcumulado)$}
  \li \For (p in particiones)
  \li   pesoViejo = p.getPeso()
  \li   pesoNuevo = pesoDelGrafoConVerticeEnParticion(p,verticeAUbicar)
      \End
  \li   \If $pesoAcumulado <= pesoGrafoConModificacion$
  \li   \Then
          $continue$
  \li   \Else
  \li     $p.setPeso(pesoNuevo)$
  \li     $pesoAcumulado = pesoGrafoConModificacion$
  \li     $combinar(particiones, proximoVertice(verticeAUbicar), pesoAcumulado)$

  \li     $pesoAcumulado = pesoGrafoConModificacion - pesoNuevo$
  \li     $p.sacarNodo(verticeAUbicar)$
  \li     $p.setPeso(pesoOld)$

        \End
\end{codebox}

\subsubsection{Podas}
La poda que aplicamos al algoritmo de BackTraking es bastante sencilla pero efectiva.\\
B\'asicamente me guardo la \'ultima mejor soluci\'on. Y si en el camino de recorrer las particiones, al agregar un nodo en alguna de las particiones, este genera una peor soluci\'on que la que ya tengo calculada, directamente la descarto y paso a intentar colocar este nodo en la proxima partici\'on.\\

En la secci\'on de experimentaci\'on vamos a mostrar los resultados de correr los algoritmos con o sin poda.






\subsection{C\'odigo Relevante}
El esta secci\'on del c\'odigo pondremos partes del c\'odigo relevante.

\lstset{language=C++,
                basicstyle=\ttfamily\footnotesize,
                keywordstyle=\color{blue}\ttfamily,
                stringstyle=\color{red}\ttfamily,
                commentstyle=\color{green}\ttfamily,
                morecomment=[l][\color{magenta}]{\#},
                breaklines=true
}
\begin{lstlisting}

class Arista {
  public:
    Arista(Vertice v, Vertice w);
    Arista(Vertice v, Vertice w, double peso);
    Vertice getVertice1();
    Vertice getVertice2();
    double getPeso();

  private:
    Vertice v;
    Vertice w;
    double peso;
};

class Grafo{
  public:
    Grafo(int n, list<Arista> aristas);
    Arista* getArista(Vertice v, Vertice w);
    Arista** getAristas(Vertice v);
    list<Arista*>* getAristas();
    int getCantVertices();

  private:
    Arista *** ady;
    list<Arista*>* aristaList;
    int n; \\ |V| := cantidad de vertices 
};

class Particion{
  public:
    Particion(int nro);
    double cuantoPesariaCon(Grafo *G, Vertice vertice);
    double cuantoPesariaSin(Grafo *G, Vertice vertice);
    void agregar(Grafo *G, Vertice vertice);
    void agregarSinActualizarPeso(Vertice vertice);
    void quitar(Grafo *G, Vertice vertice);
    void quitarUltimo(Grafo *G);
    void quitarUltimoSinActualizarPeso();
    int getNro();
    double getPeso();
    void setPeso(double peso);
    list<Vertice> getVertices();

  private:
    int nro;
    list<Vertice> vertices;
    double peso;
    double pesoConVerticeX;
    Vertice verticeX;
};

void Exacto::combinar(list<Particion> &k_particion, Vertice verticeAUbicar, double pesoAcumulado){

  //Lo siguiente equipara a decir: Si no hay mas quimicos para cargar..
  if (verticeAUbicar == G->getCantVertices()) {
    if (pesoAcumulado < this->mejorPeso){
      this->mejorPeso = pesoAcumulado;
      this->mejorKParticion = k_particion;
      if (mostrarInfo) mostrarPotencialSolucion(this->mejorKParticion, this->mejorPeso);
    }
    return;
  }

  list<Particion>::iterator itParticion;
  for (itParticion = k_particion.begin(); itParticion != k_particion.end(); itParticion++){

    double pesoOld = itParticion->getPeso();
    double pesoNew = itParticion->cuantoPesariaCon(G, verticeAUbicar);
    double difPeso = pesoNew - pesoOld;

    if (this->podaHabilitada && (this->mejorPeso <= pesoAcumulado+difPeso))
      continue;

    // 'agregar' no vuelve a calcular el peso. Ya lo calcul'o en cuantoPesariaCon donde se cachea.
    itParticion->agregar(G, verticeAUbicar);
    pesoAcumulado += difPeso;

    combinar(k_particion, verticeAUbicar+1, pesoAcumulado);

    pesoAcumulado -= difPeso;
    itParticion->quitarUltimoSinActualizarPeso();
    itParticion->setPeso(pesoOld);
  }
  

  // Si hay menos de k particiones el vertice se puede ubicar en una particion nueva.
  if (k_particion.size() < k) {
    Particion particionNueva(k_particion.size());
    particionNueva.agregar(G, verticeAUbicar);
    k_particion.push_back(particionNueva);
    //No hace falta fijarse que pesoAcumulado sea menor a mejorPeso porque el pesoAcumulado no se modifica al hacer una particion nueva sin aristas.
    combinar(k_particion, verticeAUbicar+1, pesoAcumulado);
    k_particion.pop_back();
  }

}

\end{lstlisting}


