\subsection{Descripci\'on del problema}

Nos encontramos ante una competencia que consiste en cruzar un puente dando saltos con la menor cantidad posible de estos.
Cada participante tiene una capacidad de salto o salto máximo limitado, razón por la cual no pueden cruzar el puente de un salto, a menos que su salto sea mayor a la longitud del puente. Esta longitud la medimos por la cantidad de tablones que posee el puente.
Para no hacer demasiado trivial la competencia, no todos los tablones del puente están en buen estado, algunos están rotos, pero están marcados para que no se salte a estos. Esto trae una consecuencia, puede darse el caso que el salto máximo de un participante sea n y en el puente haya m tablones rotos continuos, con m > n, por lo tanto cuando el participante llegue al tablón anterior a esta serie continua de tablones rotos, no podrá efectuar mas saltos, y pierde el juego. 

\subsection{Resoluci\'on}


\subsection{Demostraci\'on de la resoluci\'on}

\subsection{Complejidad del algoritmo}


\subsection{Codigo fuente}

%\lstset{language=C++,
%                basicstyle=\ttfamily\footnotesize,
%                keywordstyle=\color{blue}\ttfamily,
%                stringstyle=\color{red}\ttfamily,
%                commentstyle=\color{green}\ttfamily,
%                morecomment=[l][\color{magenta}]{\#},
%                breaklines=true
%}
%\begin{lstlisting}


%\end{lstlisting}

\subsection{Casos de prueba}

\subsection{Performance}
