\subsection{Descripci\'on del problema}

Nos encontramos ante una competencia que consiste en cruzar un puente dando saltos con la menor cantidad posible de estos.
Cada participante tiene una capacidad de salto o salto máximo limitado, razón por la cual no pueden cruzar el puente de un salto, a menos que su salto sea mayor a la longitud del puente. Esta longitud la medimos por la cantidad de tablones que posee el puente.
Para no hacer demasiado trivial la competencia, no todos los tablones del puente están en buen estado, algunos están rotos, pero están marcados para que no se salte a estos. Esto trae una consecuencia, puede darse el caso que el salto máximo de un participante sea n y en el puente haya m tablones rotos continuos, con m > n, por lo tanto cuando el participante llegue al tablón anterior a esta serie continua de tablones rotos, no podrá efectuar mas saltos, y pierde el juego. 

\subsection{Resoluci\'on}


\subsection{Demostraci\'on de la resoluci\'on}

\subsection{Complejidad del algoritmo}


\subsection{C\'odigo fuente}

\lstset{language=C++,
                basicstyle=\ttfamily\footnotesize,
                keywordstyle=\color{blue}\ttfamily,
                stringstyle=\color{red}\ttfamily,
                commentstyle=\color{green}\ttfamily,
                morecomment=[l][\color{magenta}]{\#},
                breaklines=true
}
\begin{lstlisting}

typedef std::vector<int> LTablonesEstado;
typedef std::vector<int> LSaltos;

LSaltos resolver(int cantTablones, int saltoMaximo, LTablonesEstado& tablones){

	LSaltos saltos;
	int contadorSaltoMaximo = 0;
	int tablonASaltar;
	int tamSalto = 0;
	bool encontroSalto = false;

	for(int nroTablon = 0; nroTablon < cantTablones; nroTablon++){

		contadorSaltoMaximo++;
		tamSalto++;
		
		//Si hay un tablon en ese lugar actualizo tablonASaltar
		if(tablones[nroTablon] == 1){
			tablonASaltar = nroTablon;
			encontroSalto = true;
		}
		
		//Si llegue al limite de salto agrego (Y pude hacer un salto) agrego el tablon al que voy a saltar a la lista de saltos y reinicio los contadores
		if(contadorSaltoMaximo == saltoMaximo && encontroSalto){			
			//Guardo el tablon al que salte
			saltos.push_back(tablonASaltar);

			//reinicio las variables para un nuevo salto
			contadorSaltoMaximo = saltoMaximo - tamSalto;
			tamSalto = 0;
			encontroSalto = false;

		}else if (contadorSaltoMaximo == saltoMaximo && !encontroSalto)
		{
			saltos.clear();
			break;
		}
	}
	return saltos;//Tambien devolver si se pudo llegar al final con saltos.
}

\end{lstlisting}

\subsection{Casos de prueba}

Para este ejercicio escogimos algunos casos de prueba que tienen las siguientes características:

\begin{itemize}

\item Caso en el que el puente no posee ningún tablón roto. Aquí el participante debería cruzar el puente dando saltos de tamaño igual a su salto máximo, ya que no hay nada que impida que no lo haga.

\item Caso en el que el puente no posee tablones sanos, y el limite de salto del jugador es menor a la cantidad de tablones del puente. El participante no podrá hacer ningún salto, ya que no importa de que longitud sea este, siempre terminará cayendo en un tablón roto.

\item Caso en el que el puente no posee tablones sanos, y el limite de salto del jugador es mayor a la cantidad de tablones del puente. En este caso el participante podrá cruzar el puente de un salto igual al tamaño del puente mas uno.

\item Caso en el que los tablones sanos están cada $x$ tablones, donde $x$ es igual al salto máximo del participante. En este caso el participante cruzara el puente en $m$ cantidad de saltos, donde $m$ es la cantidad de tablones sanos. Osea que pasará por todos los tablones sanos.

\item Caso en el cual los últimos $m$ tablones están rotos, por lo tanto el algoritmo debe ejecutarse con normalidad, pero a la hora de evaluar el ultimo salto, este no podrá efectuarse y se debe borrar la lista de saltos hechos y retornar "no".

\end{itemize}


\subsection{Performance}
