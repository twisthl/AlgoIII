
El Jefe de Desarrollo del módulo de Edificios 2D del sistema desarrollado, nos ha pedido una nueva funcionalidad para nuestro dibujador de contornos. Ahora se requiere que el algoritmo identifique aquellos edificios que no figuran en el contorno final de la ciudad y aquellos que figuran por completo en el mismo, es decir, aquellos cuya altura es la máxima altura entre todos los edificios a lo largo de todo su ancho.

\subsubsection*{Como afecta nuestro algoritmo}
\addcontentsline{toc}{subsubsection}{Como afecta nuestro algoritmo}

En nuestro algoritmo para poder identificar aquellos edificios que sobresalen de la ciudad, o sea aquellos en los cuales sus dos esquinas superiores pertenecen a la solución final, es necesario marcarlos cuando abro y cierro los edificios.
No es complejo, ya que en nuestros PuntoCritico tengo una referencia al Edificio. Con lo cual lo único que habria que agregar es algun tipo de identificación de edificio (por ejemplo un número como idEdificio, que va a ser el número de orden en la entrada de edificios) y agregar la lógica necesaria para marcar al edificio cuando corresponde.

A la hora de decidir que un punto pertenece a nuestra solución:
\begin{itemize}
	\item Si es un punto de apertura, dejo una marca de que ese edificio tiene su esquina inicial dentro de la solución final
	\item Si es uno de clausura, marco que su esquina final está en la silueta de la solución final. Y en este instante puedo verificar si ya tiene la marca de que su otra esquina también está en la solución y agregarlo en una colección de edificios que van a ser aquellos que figuran en el controno final.
\end{itemize}

Al momento de descartar algun punto no actualizo el edificio asociado con lo cual alguno de los dos puntos de su esquina no quedó marcado y por lo tanto no va a pertenecer a la colección de edificios. Más aún, puedo tener otra colección fácilmente en la cual tengo el resto de los edificios. O sea dos colecciones, Una con los que pretendo, y otra con los edificios que quedan ocultos por la silueta de la ciudad.


\subsubsection*{Que impacto tiene en la complejidad}
\addcontentsline{toc}{subsubsection}{Que impacto tiene en la complejidad}

La complejidad del algoritmo no se ve afectada ya que se realizan un par de operaciones más, al momento de decidir si un PuntoCritico agrega o no un punto a la solución, en $O(1)$ para decidir si un Edificio queda oculto en la silueta de la ciudad o no.

Y si lo quiero a la salida del algoritmo lo inserto en una lista lo cual también es $O(1)$

\lstset{language=C++,
                basicstyle=\ttfamily\footnotesize,
                keywordstyle=\color{blue}\ttfamily,
                stringstyle=\color{red}\ttfamily,
                commentstyle=\color{green}\ttfamily,
                morecomment=[l][\color{magenta}]{\#},
                breaklines=true
}
\begin{lstlisting}

struct PuntoCritico{
	PuntoCritico(bool s, Edificio& e){
		edificio = &e;
		sube = s;
		altura = e.altura;
		if (s){
			posicion = e.comienzo;
		} else {
			posicion = e.fin;
		}
	}
	bool sube;
	int altura, posicion;
	Edificio* edificio;
};

std::vector<PuntoCritico> puntos;
MapAlturas edificiosAbiertos;

std::vector<PuntoCritico>::iterator itPuntos = puntos.begin();
	//Recorro los 2N elementos
	for(;itPuntos!=puntos.end();++itPuntos){
		if (itPuntos->sube){
		//Si es un punto de comienzo de un edificio
			//agregar al mapa el edificio abierto
			edificiosAbiertos.insert(make_pair(itPuntos->altura,*itPuntos));
			std::vector<PuntoCritico>::iterator itCopy = itPuntos;
 			if(*(++itCopy) == *itPuntos){
 			//Si no es el mas alto de los edificios que empiezan en la misma posicion.	

 				/*
 				*
 				*
 					No marco la esquina del edificio, con lo cual, marque o no su otra esquina ya no pertenece a los edificios marcados.
 				*
 				*
 				*/

				continue;
			} else if (itPuntos->altura > nivelActual){
			//Si el nivel de este edificio supera al nivel actual del recorrido
				//Lo inserto en la solucion
				ciudad->push_back(itPuntos->posicion);
				ciudad->push_back(itPuntos->altura);
				//Actualizo el nivel actual
				nivelActual = itPuntos->altura;

				/*
 				*
 				*
 					Marco la esquina del edificio.
 				*
 				*
 				*/
 				marcar(itPuntos->edificio);
			}
		} else {
		//Si es un punto de cierre de un edificio
			//Sacar del mapa el edificio que estoy cerrando
			MapAlturas::iterator it = edificiosAbiertos.find(itPuntos->altura);
			edificiosAbiertos.erase (it);
			//Consigo el maximo edificio abierto
			MapAlturas::iterator itEdificiosAbiertos = --edificiosAbiertos.end();
			int alturaMaximoEdificio = (*itEdificiosAbiertos).first;
			if(itPuntos->altura == nivelActual && alturaMaximoEdificio < itPuntos->altura){
			//Si estoy bajando desde el edificio que esta abierto y no hay ninguno del mismo tamano
				//Inserto el punto de bajada hasta la posicion del maximo abierto
				ciudad->push_back(itPuntos->posicion);
				ciudad->push_back(alturaMaximoEdificio);
				//Actualizo el nivel actual
				nivelActual = alturaMaximoEdificio;
				
				/*
 				*
 				*
 					Marco la esquina del edificio. Y agrego a los edificios destacados
 				*
 				*
 				*/
				marcar(itPuntos->edificio);
				if (itPuntos->edificio->primerEsquinaMarcada){
				edificiosDestacados.add(itPuntos->edificio);
				} else{
					edificiosRestantes.add(itPuntos->edificio);
				}
			} else {
				/*
 				*
 				*
 					Agrego a los edificios que no quedan en la solucion
 				*
 				*
 				*/
				edificiosRestantes.add(itPuntos->edificio);
			}

		}
	}
\end{lstlisting}


Como podemos ver en el código lo único que se agrega es el método para \texttt{marcar} un edificio y al cerrarlo se verifica que si ya está marcado o no, lo inserta en la colección correspondiente.