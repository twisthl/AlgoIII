El algoritmo tiene 3 etapas bien definidas, las cuales vamos a detallar y calcular su complejidad.

\begin{enumerate}[I]
	\item Generar los 2N PuntoCritico a partir de los N edificios. Esto se realiza en $O(n)$ haciendo una recorrida de los edificios de entrada.
	\item Ordenar el arreglo de los 2N PuntoCritico para luego recorrer en orden. Gracias al sort\footnote{C++ reference \url{http://www.cplusplus.com/reference/algorithm/sort/}}, se obtiene en $O(nlogn)$ \footnote{$O(2n log(2n)) = O(2n (log(2)+log(n))) = O(2n+2n log(n)) = O(2n log(n)) = O(n log(n)) $}
	\item Pro último recorrer la colección de los 2N PuntoCritico para ir generando la silueta de la siudad utilizando operaciones básicas. $O(n log(n))$
	\begin{itemize}
		\item Se recorre la colección de 2N PuntoCritico; cuando es un punto de comienzo se inserta en un multimap\footnote{C++ reference \url{http://www.cplusplus.com/reference/map/multimap/}} (nos permite obtener el maximo en $O(1)$ y agregar en $O(log(n))$) un elemento con esa altura, y al analizar un punto de clausura, se lo saca del mismo. $O(log(n)$
	\end{itemize}
\end{enumerate}

Con lo cual podemos apreciar que en ningún momento se supera el orden de $O (n log(n))$, y al ser tres etapas consecutivas e independientes se puede afirmar que la complejidad es InfracCuadrática como se pide en el enunciado.