\appendix

\lstset{language=C++,
	basicstyle=\ttfamily\footnotesize,
	keywordstyle=\color{blue}\ttfamily,
	stringstyle=\color{red}\ttfamily,
	commentstyle=\color{green}\ttfamily,
	numbers=left,
	breaklines=true,
	tabsize=1,
	commentstyle=\color{magenta},
	morecomment=[l][\color{magenta}]{\#}
}


\section{Código Robanumeros} \label{App:AppendixA}

\begin{frame}

\begin{lstlisting}
typedef pair<string, uint16_t> jugada_t;

int32_t sumar(const vector<int16_t>& cartas, uint16_t i, uint16_t j) {
	int32_t suma = 0;
	for (uint16_t k = i; k <= j; ++k)
		suma += cartas[k];
	return suma;
}

void robanumeros(const vector<int16_t>& cartas, vector<jugada_t>& jugadas) {	
	uint16_t n = cartas.size();

	vector<int32_t> aux(n, 0);
	vector<vector<int32_t> > memo(n, aux);
	vector<vector<int32_t> > sumas(n, aux);

	vector<jugada_t> auxj(n, jugada_t());
	vector<vector<jugada_t> > m_jugadas(n, auxj);

	// Pre calculo las sumatorias
	for (uint16_t j = 0; j < n; ++j) {
		for (int16_t i = 0; i <= j; ++i)
			sumas[i][j] = sumar(cartas, i, j);
	}

	// Calculo de la maxima diferencia
	jugada_t jugada;
	for (uint16_t j = 0; j < n; ++j) {
		for (uint16_t i = 0; i <= j; ++i) {
			int32_t maximo = sumas[j-i][j]; // Robo to.do lo posible

			jugada = make_pair("izq", i+1);

			for (uint16_t k = 1; k <= i; ++k) {
				int32_t sumIzq = sumas[j-i][j-i+k-1];
				int32_t sumDer = sumas[j-k+1][j];

				int32_t diffIzq = (sumIzq - memo[j-i+k][j]);
				int32_t diffDer = (sumDer - memo[j-i][j-k]);

				if (diffDer < diffIzq) {
					if (maximo < diffIzq) {
						maximo = diffIzq;
						jugada = make_pair("izq", k);
					}
				}
				else if (maximo < diffDer) {
					maximo = diffDer;
					jugada = make_pair("der", k);
				}

			}
			m_jugadas[j-i][j] = jugada;
			memo[j-i][j] = maximo;
		}
	}

	// recontruyo las jugadas
	uint16_t i = 0;
	uint16_t j = n - 1;
	while (j - i + 1 > 0) {
		jugada = m_jugadas[i][j];
		jugadas.push_back(jugada);
		if (jugada.first == "izq")
			i += jugada.second;
		else
			j -= jugada.second;
	}
}

\end{lstlisting}
\end{frame}

\newpage
\section{Código La Centralita (de gas)} \label{App:AppendixB}

\begin{frame}

\begin{lstlisting}
class DisjointSet {
	public:
		DisjointSet(size_t size) {
			for (size_t i=0; i<size; i++) {
				rank.push_back(0);
				parent.push_back(i);
			}
		}

		~DisjointSet() {
			rank.clear();
			parent.clear();
		}

		size_t find(size_t x) {
			if (parent[x] != x)
				parent[x] = find(parent[x]);
			return parent[x];
		}

		void merge(size_t x, size_t y) {
			if (rank[x] > rank[y])
				parent[y] = x;
			else
				parent[x] = y;
			if (rank[x] == rank[y]) ++rank[y];
		}

	private:
		vector<size_t> rank;
		vector<size_t> parent;

};
\end{lstlisting}
\end{frame}

\newpage
\section{Código Saltos en La Matrix} \label{App:AppendixC}
\begin{frame}

\begin{lstlisting}
// Para almacenar los p de potencia de cada resorte
typedef vector<vector<uint16_t> > matrix_t;

class Vertex {
public:
	Vertex() : 
		m_i(0), 
		m_j(0), 
		m_distancia(0), 
		m_k(0), 
		m_visitado(false) 
	{}

	Vertex(uint16_t i, uint16_t j, uint16_t distancia, uint16_t k, bool visitado=true) : 
		m_i(i), 
		m_j(j), 
		m_distancia(distancia), 
		m_k(k),  
		m_visitado(visitado) 
	{}

	uint16_t m_i; // Fila
	uint16_t m_j; // Columna
	uint16_t m_distancia;
	uint16_t m_k;

	bool m_visitado; // Me visitaron?

};

class Edge {
public:
	Edge() : 
		m_i(0), 
		m_j(0), 
		m_distancia(0), 
		m_k(0)
	{}

	Edge(uint16_t i, uint16_t j, uint16_t d, uint16_t k) : 
		m_i(i), 
		m_j(j), 
		m_distancia(d), 
		m_k(k) 
	{}

	uint16_t m_i; // Fila
	uint16_t m_j; // Columna

	uint16_t m_distancia; // Distancia
	uint16_t m_k; // El k acumulado!!! Ojo con esto.
};

void saltoMatrix(matrix_t& matrix, const Vertex& origen, const Vertex& destino, uint16_t k, list<Vertex>& saltos) {
	uint16_t n = matrix.size();

	vector<Vertex> aux(k+1, Vertex());
	vector<vector<Vertex> > aux2(n, aux);
	vector<vector<vector<Vertex> > > E(n, aux2);

	queue<Edge> cola; // La cola de resortes a visitar

	cola.push( Edge(origen.m_i, origen.m_j, 0, 0) );
	E[origen.m_i][origen.m_j][0].m_visitado = true;
	E[origen.m_i][origen.m_j][0].m_distancia = 0;

	while (!cola.empty()) {
		Edge u = cola.front(); cola.pop(); // Saco el proximo de la cola

		uint16_t potencia = matrix[u.m_i][u.m_j];

		// Para en la misma columna, salto a las filas que puedo. Hacia arriba y hacia abajo
		for(int16_t i = u.m_i - potencia; i <= u.m_i + potencia; ++i) {
			if( (i >= 0) && (i < n)) {
				if(!E[i][u.m_j][u.m_k].m_visitado) {
					E[i][u.m_j][u.m_k] = Vertex(u.m_i, u.m_j, u.m_distancia+1, u.m_k); // se marca como visitado por default
					cola.push( Edge(i, u.m_j, u.m_distancia+1, u.m_k));
				}
			}
		}

		// Para en la misma fila, salto a las columnas que puedo. Hacia la izquierda y hacia la derecha
		for(int16_t j = u.m_j - potencia; j <= u.m_j + potencia; ++j) {
			if( (j >= 0) && (j < n)) {
				if(!E[u.m_i][j][u.m_k].m_visitado) {
					E[u.m_i][j][u.m_k] = Vertex(u.m_i, u.m_j, u.m_distancia+1, u.m_k); // se marca como visitado por default
					cola.push( Edge(u.m_i, j, u.m_distancia + 1, u.m_k) );
				}
			}
		}

		// Ahora a usar los k extras que tengo
		for(int16_t s = 1; s <= k - u.m_k; ++s) {
			// En la misma columna salto a la fila de arriba
			{
				int16_t i = u.m_i - potencia - s;
				if( (i >= 0) && (i < n)) {
					if(!E[i][u.m_j][u.m_k + s].m_visitado) {
						E[i][u.m_j][u.m_k + s] = Vertex(u.m_i, u.m_j, u.m_distancia+1, u.m_k); // se marca como visitado por default
						cola.push( Edge(i, u.m_j, u.m_distancia + 1, u.m_k + s));
					}
				}
			}

			{ // En la misma columna salto a la fila de abajo
				int16_t i = u.m_i + potencia + s;
				if( (i >= 0) && (i < n)) {
					if(!E[i][u.m_j][u.m_k + s].m_visitado) {
						E[i][u.m_j][u.m_k + s] = Vertex(u.m_i, u.m_j, u.m_distancia+1, u.m_k); // se marca como visitado por default
						cola.push( Edge(i, u.m_j, u.m_distancia + 1, u.m_k + s));
					}
				}

			}

			{ // En la misma fila salto a la columna de la izquierda
				int16_t j = u.m_j - potencia - s;
				if ((j >= 0) && (j < n)) {
					if(!E[u.m_i][j][u.m_k + s].m_visitado) {
						E[u.m_i][j][u.m_k + s] = Vertex(u.m_i, u.m_j, u.m_distancia+1, u.m_k); // se marca como visitado por default
						cola.push( Edge(u.m_i, j, u.m_distancia+1, u.m_k + s));
					}
				}

			}

			{ // En la misma fila salto a la columna de la derecha
				int16_t j = u.m_j + potencia + s;
				if ((j >= 0) && (j < n)) {
					if(!E[u.m_i][j][u.m_k + s].m_visitado) {
						E[u.m_i][j][u.m_k + s] = Vertex(u.m_i, u.m_j, u.m_distancia+1, u.m_k); // se marca como visitado por default
						cola.push( Edge(u.m_i, j, u.m_distancia+1, u.m_k + s));
					}
				}

			}
		}
	}

	// el mejor salto siempre no deberia estar siempre en k? PREGUNTAR!!!
	Edge elMejorSalto = Edge(destino.m_i, destino.m_j, E[destino.m_i][destino.m_j][0].m_distancia, 0);
	for(uint16_t s = 1; s <= k; ++s) {
		if(elMejorSalto.m_distancia > E[destino.m_i][destino.m_j][s].m_distancia) {
			elMejorSalto = Edge(destino.m_i, destino.m_j, E[destino.m_i][destino.m_j][s].m_distancia, s);
		}
	}

	// Construyo el camino de saltos a realizar ( ojo con el K, esta acumulado)
	Vertex salto = Vertex(elMejorSalto.m_i, elMejorSalto.m_j, elMejorSalto.m_distancia, elMejorSalto.m_k);
	while (salto.m_i != origen.m_i || salto.m_j != origen.m_j ) {
		saltos.push_front(salto);
		salto = E[salto.m_i][salto.m_j][salto.m_k];
	}
}

\end{lstlisting}
\end{frame}

